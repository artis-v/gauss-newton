\documentclass[12pt, a4paper]{article}
\usepackage[a4paper, margin=3.8cm]{geometry}
\usepackage[utf8]{inputenc}

\usepackage{amsmath}
\usepackage{amsfonts}

\usepackage{hyperref}
\hypersetup{colorlinks=true}

\usepackage{fancyvrb}

\usepackage{array}

\usepackage{listingsutf8}
\usepackage[apple]{modernlistings} % Can be found at https://gist.github.com/maelquerre/b625bda6cab11c1b0ae67a2be86f027e

\setlength\parindent{0pt}
\setlength\parskip{10pt}

\newcommand{\bx}{{\boldsymbol{x}}}
\newcommand{\bd}{{\boldsymbol{\delta}}}

\title{Project report:
\\ Gauss-Newton optimization method}

\author{Nicolas Munke Cilano
\\ Vidar Gimbringer 
\\ Artis Vijups 
\vspace{20pt} \\ Supervisor: Stefan Diehl}
% OBS: Don't publish personal numbers to GitHub!

\begin{document}

\maketitle

\section{Introduction}

This project investigates fitting a function such as \[\varphi(\bx; t)=x_1e^{-x_2t}+x_3e^{-x_4t},\] where $\bx={(x_1,x_2,x_3,x_4)}^{T}$ are parameters, to a given set of data points $(t_i,y_i), i=1,\ldots,m$.

In particular, we seek to minimize the sum of the squared distances from each data point $(t_i,y_i)$ to the point $(t_i,\varphi(\bx; t_i))$. If we denote the distance by $r_i(\bx)=\varphi(\bx; t_i)-y_i$, then our goal is to \[\underset{\bx\in\mathbb{R}^4}{\text{minimize}}~f(\bx)\quad\text{where}\quad f(\bx)=\sum_{i=1}^{m}{r_i(\bx)}^2.\]

To solve this minimization problem, we use the Gauss-Newton method.

\section{Methods}

\subsection{Gauss-Newton method}

Let $r(\bx)={(r_1(\bx),\ldots,r_m(\bx))}^T$. We claim that \[r(\bx+\bd)\approx r(\bx)+J(\bx)\bd,\]where $J(\bx)$ is the Jacobian matrix of $r(\bx)$ and $\bd$ is an increment vector, also a direction vector.

This can be thought of as extending the concept of using the tangent line to estimate nearby values to higher dimensions.

We then observe that \begin{align*}
f(\bx+\bd) &= \sum_{i=1}^{m}{r_i(\bx+\bd)}^2 \\ 
&= {r(\bx+\bd)}^T r(\bx+\bd) \\
&\approx {(r(\bx)+J(\bx)\bd)}^T (r(\bx)+J(\bx)\bd).
\end{align*}

Since we want the output of $f $ to be the minimum possible, the gradient of this approximation should be $0$. Writing that as an equation and simplifying, we end up with \[-{J(\bx)}^T J(\bx)\bd={J(\bx)}^T r(\bx).\]

This presents the following iterative algorithm. First, we make an initial guess $\bx_0$ for $\bx$, and then we:
\begin{enumerate}
    \item Solve the linear system $-{J(\bx)}^T J(\bx)\bd={J(\bx)}^T r(\bx)$ for $\bd$.
    \item Determine an optimal step length $\lambda$ for direction $\bd$ using a line search algorithm.
    \item Update $\bx$ to $\bx+\lambda\bd$.
\end{enumerate}
We repeat these actions until the step $\lambda\bd$ is smaller than some chosen tolerance. Step size tolerance is a good choice for the stop criterion because we already have to compute $\lambda\bd$ as part of each iteration.

As a practical consideration, we have added a small regularization term to the linear system in our implementation, as in \[-({J(\bx)}^T J(\bx)+\varepsilon I)\bd={J(\bx)}^T r(\bx),\] to help prevent numerical instability.

\subsection{Line search algorithm}

For the line search mentioned in the second step of the Gauss-Newton iteration algorithm, we use Armijo's rule on $F(\lambda)=f(\bx+\lambda\bd)$.

Let $T(\lambda)=F(0)+\varepsilon F'(0)\lambda$ be a straight line through $(0,F(0))$ with less negative slope than the point's tangent, so $0<\varepsilon<1$.

Armijo's rule is made up of two (upper and lower) conditions, which are\[F(\lambda)\le T(\lambda)\quad\text{and}\quad F(\alpha\lambda)\ge T(\alpha\lambda)\text{ for fixed }\alpha>1.\]
This rule ensures $\lambda$ will be in an \textit{interval} of points where $F $ is substantially smaller. Computationally, this is faster than looking for a perfect choice for $\lambda$. This makes an Armijo's rule based algorithm a good choice for this project.

So that $\lambda$ satisfies Armijo's rule, we choose it as follows:\begin{enumerate}
    \item Make an initial guess for $\lambda$.
    \item Repeatedly scale $\lambda$ up by $\alpha$ until it satisfies the lower condition.
    \item Repeatedly scale $\lambda$ down by $\alpha$ until it satisfies the upper condition. 
\end{enumerate}

\section{Project work}

\subsection{Structure}

The project is \href{https://github.com/artis-v/gauss-newton}{stored on GitHub as a repository}. It is made up of:\begin{itemize}
    \item phi1.m, phi2.m, data1.m, data2.m, grad.m as provided,
    \item gaussnewton.m, the implementation of the Gauss-Newton method,
    \item line\_search.m, the line search algorithm based on Armijo's rule,
    \item script.m, the main script containing tests and tasks,
    \item report.tex and report.pdf, forming this report,
    \item metadata and configuration files.
\end{itemize}

\subsection{Responsibilities}

Nicolas handled most of the MATLAB programming for the project, including implementing the Gauss-Newton method and creating the main script file.

Artis was responsible for maintaining the project repository, and also helped with implementing the line search algorithm in MATLAB.

Vidar and Artis worked together on creating the $\textit{\LaTeX}$ report for the project. In particular, Vidar worked on the analysis of different initial guesses and strategies for choosing the initial point.

\subsection{Results}

We fit the two functions from phi1.m and phi2.m, \[\varphi_1(\bx;t)=x_1e^{-x_2t},\quad\bx={(x_1,x_2)}^T\] and \[\varphi_2(\bx;t)=\varphi(\bx; t)=x_1e^{-x_2t}+x_3e^{-x_4t},\quad\bx={(x_1,x_2,x_3,x_4)}^T,\] to the two sets of data points \texttt{data1} and \texttt{data2} from data1.m and data2.m. Throughout, $x_n=n$ is used for the initial guess. The results are:
\begin{center}\bgroup\def\arraystretch{1.4}
\begin{tabular}%
  {|>{\raggedright\arraybackslash}p{1.8cm}%
    >{\raggedright\arraybackslash}p{6cm}%
    >{\centering\arraybackslash}p{2.3cm}%
    >{\centering\arraybackslash}p{1.5cm}|}
    \hline
    Case & Optimal point $\bx$ & $\Vert \nabla f(\bx) \Vert_2$ & $\Vert r(\bx) \Vert_{\infty}$
    \\ \hline
    \texttt{data1}, $\varphi_1$ & ${(10.8108, 2.4786)}^T$ & $3.7450\cdot10^{-4}$ & $1.6287$
    \\
    \texttt{data2}, $\varphi_1$ & ${(12.9789, 1.7861)}^T$ & $8.0031\cdot10^{-2}$ & $1.0397$
    \\
    \texttt{data1}, $\varphi_2$ & ${(6.3445, 10.5866, 6.0959, 1.4003)}^T$ & $4.0651\cdot10^{-5}$ & $0.4334$
    \\
    \texttt{data2}, $\varphi_2$ & ${(4.1741, 0.8747, 9.7390, 2.9208)}^T$ & $9.5220\cdot10^{-3}$ & $0.1182$
    \\ \hline
\end{tabular}
\egroup\end{center}

\subsection{Initial points}

We consider multiple initial guesses for the \texttt{data1}, $\varphi_2$ case, with the tolerance set to $10^{-4}$.

It is easily seen that $x_2$ and $x_4$ must be non-negative for this function, as a negative value causes the function to blow up. The considered initial guesses follow this restriction.

We use the number of iterations and function evaluations to define how fast the system converges. 

\begin{center}\bgroup\def\arraystretch{1.4}
{\small
\begin{tabular}%
  {|>{\raggedright\arraybackslash}p{2cm}%
    >{\raggedleft\arraybackslash}p{2cm}%
    >{\raggedright\arraybackslash}p{2cm}%
    w{r}{5.6cm}|}
  \hline
  Initial guess $\bx_0$ & Total iterations & Function evaluations & \vtop{\hbox{\strut{Optimal}}\hbox{\strut{ point $\bx$}}} \\ 
  \hline
  ${(0,0,0,0)}^T$ & $15$ & $49$ & ${(6.3446, 10.5865, 6.0959, 1.4003)}^T$ \\
  ${(100, 100, 100, 100)}^T$ & $14$ & $80$ & ${(6.0959, 1.4003, 6.3445, 10.5865)}^T$ \\
  ${(10, 10, 10, 10)}^T$ & $11$ & $64$ & ${(6.0959, 1.4003, 6.3445, 10.5866)}^T$ \\
  ${(1, 1, 1, 1)}^T$ & $8$ & $49$ & ${(6.0959, 1.4003, 6.3445, 10.5867)}^T$ \\
  ${(-10, 0, -10, 0)}^T$ & $21$ & $81$ & ${(6.3446, 10.5864, 6.0958, 1.4003)}^T$ \\
  ${(10, 0, 10, 0)}^T$ & $16$ & $59$ & ${(6.3445, 10.5866, 6.0959, 1.4003)}^T$ \\
  ${(-5, 0, -5, 0)}^T$ & $11$ & $53$ & ${(6.3445, 10.5865, 6.0959, 1.4003)}^T$ \\
  ${(5, 0, 5, 0)}^T$ & $11$ & $54$ & ${(6.3446, 10.5864, 6.0959, 1.4003)}^T$ \\
  ${(-1, 0, -1, 0)}^T$ & $16$ & $98$ & ${(6.0959, 1.4003, 6.3446, 10.5864)}^T$ \\
  ${(1, 0, 1, 0)}^T$ & $12$ & $60$ & ${(6.3446, 10.5863, 6.0958, 1.4003)}^T$ \\
  ${(-1, 0, -5, 0)}^T$ & $23$ & $296$ & ${(3.3205, 4013.4884, 9.1190, 3631.4299)}^T$ \\
  \hline
\end{tabular}
}
\egroup\end{center}

The most efficient initial guess with the fastest convergence appears to be \[\bx_0={(1,1,1,1)}^T.\]

\subsection{Initial guess strategy}

We introduce a strategy for picking the initial guess for $\varphi_2.$ 

To find an accurate initial guess for \[\varphi_2(\bx;t)=x_1e^{-x_2t}+x_3e^{-x_4t},\] we may make use of \[\varphi_1(\bx;t)=x_1e^{-x_2t}.\] 

Comparing these two functions, we can use prior results from fitting $\varphi_1$ in order to get a good initial guess for $\varphi_2$. By running the program multiple times to fit $\varphi_1$ with \texttt{data2}, we can get initial values for $x_1$ and $x_2$ that we can then use in the \texttt{data2}, $\varphi_2$ case.

As we have $-x_4$ in the exponential, we can restrict $x_4$ values to being non-negative. Meanwhile, $x_3$ still needs to be sought through a range of values, based on, for example, minimizing the amount of iterations or function evaluations.

\subsection{Detailed result}

For the \texttt{data2}, $\varphi_2$ case, we present a full printout. The initial point for this example is ${(1,2,3,4)}^T$.
\scriptsize\begin{Verbatim}[frame=single]
  Gauss-Newton Iteration 1: Starting line search...
    Line Search Backtrack Iteration: lambda = 5.0000e-01, F(lambda) = 2.8163e+22
    Line Search Backtrack Iteration: lambda = 2.5000e-01, F(lambda) = 8.4145e+10
    Line Search Backtrack Iteration: lambda = 1.2500e-01, F(lambda) = 1.5305e+05
    Line Search Backtrack Iteration: lambda = 6.2500e-02, F(lambda) = 2.7764e+04
  Gauss-Newton Iteration 1:
  x = [-0.1860, 0.3882, 4.7977, 3.2363]
  max(abs(r)) = 1.0019e+01, norm(grad) = 6.8024e+03, lambda = 6.2500e-02
  Gauss-Newton Iteration 2: Starting line search...
    Line Search Backtrack Iteration: lambda = 5.0000e-01, F(lambda) = 2.7633e+06
    Line Search Backtrack Iteration: lambda = 2.5000e-01, F(lambda) = 1.4590e+04
  Gauss-Newton Iteration 2:
  x = [0.7098, -0.8628, 6.2213, 2.9969]
  max(abs(r)) = 9.4068e+00, norm(grad) = 1.1161e+04, lambda = 2.5000e-01
  Gauss-Newton Iteration 3: Starting line search...
  Gauss-Newton Iteration 3:
  x = [0.8378, -0.3380, 12.8793, 1.4757]
  max(abs(r)) = 7.0874e+00, norm(grad) = 1.5113e+04, lambda = 1.0000e+00
  Gauss-Newton Iteration 4: Starting line search...
  Gauss-Newton Iteration 4:
  x = [3.4844, 1.2798, 10.2036, 2.5278]
  max(abs(r)) = 2.3072e+00, norm(grad) = 2.1603e+04, lambda = 1.0000e+00
  Gauss-Newton Iteration 5: Starting line search...
  Gauss-Newton Iteration 5:
  x = [1.6255, 0.2138, 12.2776, 2.7719]
  max(abs(r)) = 6.2761e-01, norm(grad) = 1.5974e+03, lambda = 1.0000e+00
  Gauss-Newton Iteration 6: Starting line search...
  Gauss-Newton Iteration 6:
  x = [2.6973, 0.7203, 11.1893, 2.7070]
  max(abs(r)) = 5.4273e-01, norm(grad) = 9.6006e+02, lambda = 1.0000e+00
  Gauss-Newton Iteration 7: Starting line search...
    Line Search Iteration: lambda = 2.0000e+00, F(lambda) = 2.0939e+01
  Gauss-Newton Iteration 7:
  x = [5.5035, 1.1028, 8.4286, 3.0652]
  max(abs(r)) = 3.2980e-01, norm(grad) = 8.9192e+02, lambda = 2.0000e+00
  Gauss-Newton Iteration 8: Starting line search...
  Gauss-Newton Iteration 8:
  x = [3.7177, 0.8468, 10.1922, 2.8388]
  max(abs(r)) = 2.4740e-01, norm(grad) = 2.7354e+02, lambda = 1.0000e+00
  Gauss-Newton Iteration 9: Starting line search...
  Gauss-Newton Iteration 9:
  x = [4.1808, 0.8779, 9.7318, 2.9178]
  max(abs(r)) = 1.9147e-01, norm(grad) = 3.4489e+02, lambda = 1.0000e+00
  Gauss-Newton Iteration 10: Starting line search...
  Gauss-Newton Iteration 10:
  x = [4.1741, 0.8747, 9.7390, 2.9208]
  max(abs(r)) = 1.2141e-01, norm(grad) = 7.5698e+00, lambda = 1.0000e+00
  Gauss-Newton Iteration 11: Starting line search...
    Line Search Iteration: lambda = 2.0000e+00, F(lambda) = 8.9616e+00
  Gauss-Newton Iteration 11: Converged with tol 1.0000e-04, max(abs(r)) 1.1817e-01.
  
  Final Results:
  x = [4.1741, 0.8747, 9.7390, 2.9208]
  Final norm(grad) = 9.5220e-03
  Total iterations = 11
  Total function evaluations = 30
  Total elapsed time: 0.2712 seconds
\end{Verbatim}

\pagebreak

\pagenumbering{gobble}

\section*{Appendix}

\subsection*{gaussnewton.m}
\lstinputlisting[language=Matlab,
numbers=none, 
xleftmargin=0pt,
numbersep=0pt,
basicstyle=\ttfamily\scriptsize]{gaussnewton.m}

\pagebreak

\subsection*{line\_search.m}
\lstinputlisting[language=Matlab,
numbers=none, 
xleftmargin=0pt,
numbersep=0pt, 
basicstyle=\ttfamily\scriptsize]{line_search.m}

\pagebreak

\subsection*{script.m}
\lstinputlisting[language=Matlab,
numbers=none, 
xleftmargin=0pt,
numbersep=0pt, 
basicstyle=\ttfamily\scriptsize]{script.m}

\end{document}