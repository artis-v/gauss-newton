\documentclass[12pt, a4paper]{article}
\usepackage[a4paper, margin=4cm]{geometry}

\usepackage{amsmath}
\usepackage{amsfonts}

\setlength\parindent{0pt}
\setlength\parskip{10pt}

\newcommand{\bx}{{\boldsymbol{x}}}
\newcommand{\bd}{{\boldsymbol{\delta}}}

\title{Project report:
\\ Gauss-Newton optimization method}

\author{Nicolas Munke Cilano
\\ Vidar Gimbringer 
\\ Artis Vijups 
\vspace{20pt} \\ Supervisor: Stefan Diehl}
% OBS: Don't publish personal numbers to GitHub!

\begin{document}

\maketitle

\section{Introduction}

In this project, we mainly investigate the problem of fitting the function \[\varphi(\bx; t)=x_1e^{-x_2t}+x_3e^{-x_4t},\] where $\bx={(x_1,x_2,x_3,x_4)}^{T}$ are parameters, to a given set of data points $(t_i,y_i), i=1,\ldots,m$.

In particular, we seek to minimize the sum of the squared distances from each data point $(t_i,y_i)$ to the point $(t_i,\varphi(\bx; t_i))$. If we denote the distance by $r_i(\bx)=\varphi(\bx; t_i)-y_i$, then our goal is to \[\underset{\bx\in\mathbb{R}^4}{\text{minimize}}~f(\bx)\quad\text{where}\quad f(\bx)=\sum_{i=1}^{m}{r_i(\bx)}^2.\]

To solve this minimization problem, we use the Gauss-Newton method.

\section{Methods}

\subsection{Gauss-Newton method}

Let $r(\bx)={(r_1(\bx),\ldots,r_m(\bx))}^T$. The method is set up on the idea that \[r(\bx+\bd)\approx r(\bx)+J(\bx)\bd,\]where $J(\bx)$ is the Jacobian matrix of $r(\bx)$ and $\bd$ is an increment vector, also a direction vector.

This can be thought of as an extension to more dimensions of the strategy of moving along the tangent line to estimate a nearby value.

We then observe that \begin{align*}
f(\bx+\bd) &= \sum_{i=1}^{m}{r_i(\bx+\bd)}^2 \\ 
&= {r(\bx+\bd)}^T r(\bx+\bd) \\
&\approx {(r(\bx)+J(\bx)\bd)}^T (r(\bx)+J(\bx)\bd).
\end{align*}

Since we want the output of $f $ to be the minimum possible, the gradient of this approximation should be $0$. Writing that as an equation and simplifying, we end up with \[{J(\bx)}^T J(\bx)\bd=-{J(\bx)}^T r(\bx).\]

This presents the following iterative algorithm:
\begin{enumerate}
    \item Solve the linear system ${J(\bx)}^T J(\bx)\bd=-{J(\bx)}^T r(\bx)$ for $\bd$.
    \item Determine an optimal step length $\lambda$ for direction $\bd$ using a line search algorithm.
    \item Update $\bx$ to $\bx+\lambda\bd$.
\end{enumerate}
We repeat these actions until the step $\lambda\bd$ is smaller than our chosen tolerance.

\subsection{Line search algorithm}

For the line search mentioned in the second step of the Gauss-Newton iteration algorithm, we use Armijo's rule on $F(\lambda)=f(\bx+\lambda\bd)$.

Let $T(\lambda)=F(0)+\varepsilon F'(0)\lambda$ be a straight line through $(0,F(0))$ with less negative slope than the point's tangent, so $0<\varepsilon<1$.

Armijo's rule is made up of two (upper and lower) conditions, which are\[F(\lambda)\le T(\lambda)\quad\text{and}\quad F(\alpha\lambda)\ge T(\alpha\lambda)\text{ for fixed }\alpha>1.\]
This rule ensures $\lambda$ will be in an \textit{interval} of points where $F $ is substantially smaller. Computationally, this is faster than looking for a perfect choice for $\lambda$.

So that $\lambda$ satisfies Armijo's rule, we choose it as follows:\begin{enumerate}
    \item Make an initial guess for $\lambda$.
    \item Repeatedly scale $\lambda$ up by $\alpha$ until it satisfies the lower condition.
    \item Repeatedly scale $\lambda$ down by $\alpha$ until it satisfies the upper condition. 
\end{enumerate}

\section{Project work}

% See instructions.

\subsection{Responsibilities}

% Who did what, basically.

\section*{Appendix}

% Our program should go in the appendix.

\end{document}